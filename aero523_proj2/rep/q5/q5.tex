\pagebreak
\subsection{Adaptive Iterations}
%[15\%] Perform adaptive iterations for $\alpha = [0.5, 1, 1.5, 2, 2.5, 3]$ degrees. Run the same number of adaptive iterations for each $\alpha$ at least 5. Plot the ATPR output from your finest mesh versus alpha, and discuss the trend. Include flowfield plots to augment your discussion.

In the final section, I will vary the angle of attack as well as performing adaptive mesh refinements to determine the effects of $\alpha$ on the Mach field plot, the total pressure field plot, and the ATPR output. 

\subsubsection{ATPR Versus Angle of Attack}
\begin{figure}[h]
    \centering
    \includegraphics[width = 0.9\linewidth]{rep/q5/ATPR.pdf}
    \caption[ATPR and Angle of Attack]{Effects of varying $\alpha$ on the ATPR output.}
    \label{fig:aoa_ATPR}
\end{figure}

Shown above in Figure \ref{fig:aoa_ATPR}, is the effect of varying the angle of attack on ATPR output. As shown from the Figure above, the ATPR increases with the angle of attack $\alpha$ since at an angled attack the flow enters the engine and allows for less loss of total pressure through the shock trains along the interior of the engine. Should in Figure \ref{fig:mach05} the first shock leading into the combustor hits just past the corner of the leading edge resulting in a reflected shock resulting in larger losses through the interior of the engine. Comparatively in Figure \ref{fig:mach30}, the first shock leads directly into this corner and lines to the natural compression fan along the top resulting in a lower loss of total pressure. These are subtle differences shown in Figure \ref{fig:mach_fields}, but are most visible by ``\textit{blurred}'' Mach lines along the interior of the engine where the shock trains occur.

\pagebreak
\subsubsection{Flow Fields for Varying Angle of Attacks}
\begin{figure}[h]
    \centering
    \begin{subfigure}[h]{0.48\linewidth}
        \centering
        \includegraphics[width=\linewidth]{rep/q5/mach_a5.pdf}
        \caption{Mach field at $\alpha=0.5\degree$.}\label{fig:mach05}
    \end{subfigure}
    \begin{subfigure}[h]{0.48\linewidth}
        \centering
        \includegraphics[width=\linewidth]{rep/q5/mach_a10.pdf}
        \caption{Mach field at $\alpha=1.0\degree$.}
    \end{subfigure}

    \begin{subfigure}[h]{0.48\linewidth}
        \centering
        \includegraphics[width=\linewidth]{rep/q5/mach_a15.pdf}
        \caption{Mach field at $\alpha=1.5\degree$.}
    \end{subfigure}
    \begin{subfigure}[h]{0.48\linewidth}
        \centering
        \includegraphics[width=\linewidth]{rep/q5/mach_a20.pdf}
        \caption{Mach field at $\alpha=2.0\degree$.}
    \end{subfigure}

    \begin{subfigure}[h]{0.48\linewidth}
        \centering
        \includegraphics[width=\linewidth]{rep/q5/mach_a25.pdf}
        \caption{Mach field at $\alpha=2.5\degree$.}
    \end{subfigure}
    \begin{subfigure}[h]{0.48\linewidth}
        \centering
        \includegraphics[width=\linewidth]{rep/q5/mach_a30.pdf}
        \caption{Mach field at $\alpha=3.0\degree$.}\label{fig:mach30}
    \end{subfigure}
    \caption[Mach Field with Varying Angle of Attack]{Varying angle of attack, and its effect on the mach field.}
    \label{fig:mach_fields}
\end{figure}

\paragraph{Effect on Mach Field from Varying Angle of Attack} Shown above in Figure \ref{fig:mach_fields} are the Mach fields for varying angles of attack. As shown above, there is not a large difference between the angle of attack configurations, but the main differences can be seen inside the interior of the engine. Within the interior of the engine we can see the shock trains lessing in effect resulting in a lower loss in total pressure that can improve the ATPR.

\pagebreak
\begin{figure}[h]
    \centering
    \begin{subfigure}[h]{0.48\linewidth}
        \centering
        \includegraphics[width=\linewidth]{rep/q5/pt_a5.pdf}
        \caption{Total pressure field at $\alpha=0.5\degree$.}
    \end{subfigure}
    \begin{subfigure}[h]{0.48\linewidth}
        \centering
        \includegraphics[width=\linewidth]{rep/q5/pt_a10.pdf}
        \caption{Total pressure field at $\alpha=1.0\degree$.}
    \end{subfigure}

    \begin{subfigure}[h]{0.48\linewidth}
        \centering
        \includegraphics[width=\linewidth]{rep/q5/pt_a15.pdf}
        \caption{Total pressure field at $\alpha=1.5\degree$.}
    \end{subfigure}
    \begin{subfigure}[h]{0.48\linewidth}
        \centering
        \includegraphics[width=\linewidth]{rep/q5/pt_a20.pdf}
        \caption{Total pressure field at $\alpha=2.0\degree$.}
    \end{subfigure}

    \begin{subfigure}[h]{0.48\linewidth}
        \centering
        \includegraphics[width=\linewidth]{rep/q5/pt_a25.pdf}
        \caption{Total pressure field at $\alpha=2.5\degree$.}
    \end{subfigure}
    \begin{subfigure}[h]{0.48\linewidth}
        \centering
        \includegraphics[width=\linewidth]{rep/q5/pt_a30.pdf}
        \caption{Total pressure field at $\alpha=3.0\degree$.}
    \end{subfigure}
    \caption[Total Pressure Field with Varying Angle of Attack]{Varying angle of attack, and its effect on the total pressure field.}
    \label{fig:pt_fields}
\end{figure}

\paragraph{Effect on Total Pressure Field from Varying Angle of Attack} Shown above in Figure \ref{fig:pt_fields} are the total pressure fields for varying angles of attack. In this analysis, there is not a large difference (at least not as much as the Mach field), which alters the total pressure within the nozzle. 

However, the most notable differences will be the total pressure below the engine and at the exit of the engine. Underneath the engine the difference is caused by a larger stagnation of pressure due to the increased angle of attack. At the exit of the engine for higher angles of attack the total pressure losses will be lower resulting in a higher total pressure than if there were more losses due to the shocks.


