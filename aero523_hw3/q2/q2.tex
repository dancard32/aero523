\pagebreak
\section{The Beam-Warming Method}
Consider the Beam-Warming (BW) method applied to the one-dimensional advecton equation, $u_t + au_x= 0,\ a >0$, with initial condition $u(x,0) =u_0(x),\ x \in [0,L]$ and periodic boundaries.

\begin{enumerate}[label=\alph*., start = 1]
    \item Derive the modified equation for the BW method and express it in the form
    
    \vspace{-0.2in}
    \begin{equation*}
        u_t + au_x = \alpha u_{xx} - \beta u_{xxx}
    \end{equation*}
    Use this equation to determine the order of accuracy of the BW method, and discuss the dispersion relation.

    \vspace{-0.35in}
    \begin{align*}
        \shortintertext{Starting with the modified equation for Beam-Warming method,}
        u_j^{n+1} & = u_j^n - \frac{\sigma}{2}\left(3u_j^n - u_{j-1}^n + u_{j-2}^n\right) + \frac{\sigma^2}{2}\left(u_{j-2}^n - 2u_{j-1}^n + u_j^n\right)\\
        \shortintertext{Conducting the Taylor series expansions for these nodes gives,}
        u_{j-2}^n & = u_j^n - 2\Delta xu_x + 2\Delta x^2 u_{xx} - \frac{4}{3}\Delta x^3u_{xxx} + \frac{2}{3}\Delta x^4u_{x^{(4)}} + \ldots \mathcal{O}(\Delta x^5)\\
        u_{j-1}^n & = u_j^n - \Delta x u_x + \frac{1}{2}\Delta x^2 u_{xx} - \frac{1}{6}\Delta x^3 u_{xxx} + \frac{1}{24}\Delta x^4 u_{x^{(4)}} + \ldots \mathcal{O}(\Delta x^5)\\
        u_{j}^{n+1} & = u_j^n + \Delta tu_t + \frac{1}{2}\Delta t^2u_{tt} + \frac{1}{6}\Delta t^3 u_{ttt} + \frac{1}{24}\Delta t^4u_{t^{(4)}} + \ldots \mathcal{O}(\Delta t^5)\\
        \shortintertext{Expanding the right-hand side of the expression I get,}
        \text{RHS} & = u_j^n - \frac{\sigma}{2}\left(3u_j^n - 4u_{j-1}^n + u_{j-2}^n\right) + \frac{\sigma^2}{2}\left(u_{j-2}^n - 2u_{j-1}^n + u_j^n\right)\\
        \shortintertext{Expressing each quantity I get,}
        3u_j^n - 4u_{j-1}^n + u_{j-2}^n & = 3u_j^n + \ldots\\
        & - 4\left(u_j^n - \Delta x u_x + \frac{1}{2}\Delta x^2 u_{xx} - \frac{1}{6}\Delta x^3 u_{xxx} + \frac{1}{24}\Delta x^4 u_{x^{(4)}} + \ldots \mathcal{O}(\Delta x^5)\right) + \ldots\\
        & + u_j^n - 2\Delta xu_x + 2\Delta x^2 u_{xx} - \frac{4}{3}\Delta x^3u_{xxx} + \frac{2}{3}\Delta x^4u_{x^{(4)}} + \ldots \mathcal{O}(\Delta x^5)\\
        & = 2\Delta xu_x - \frac{2}{3}\Delta x^3 u_{xxx} + \frac{1}{2}\Delta x^4u_{x^{(4)}} + \mathcal{O}(\Delta x^5)\\
        u_{j-2}^n - 2u_{j-1}^n + u_j^n & = u_j^n - 2\Delta xu_x + 2\Delta x^2 u_{xx} - \frac{4}{3}\Delta x^3u_{xxx} + \frac{2}{3}\Delta x^4u_{x^{(4)}} + \ldots \mathcal{O}(\Delta x^5) + \ldots \\
        & -2\left(u_j^n - \Delta x u_x + \frac{1}{2}\Delta x^2 u_{xx} - \frac{1}{6}\Delta x^3 u_{xxx} + \frac{1}{24}\Delta x^4 u_{x^{(4)}} + \ldots \mathcal{O}(\Delta x^5)\right) + \ldots \\
        & + u_j^n\\
        & = \Delta x^2 u_{xx} - \Delta x^3 u_{xxx} + \frac{7}{12}\Delta x^4 u_{x^{(4)}} - \frac{1}{4}\Delta x^5u_{x^{(5)}} + \mathcal{O}(\Delta x^6)\\
        \shortintertext{Setting up the relationships I get that,}
        u_j^n + \Delta tu_t + \frac{1}{2}\Delta t^2u_{tt} + & \frac{1}{6}\Delta t^3 u_{ttt} + \frac{1}{24}\Delta t^4u_{t^{(4)}} + \ldots \mathcal{O}(\Delta t^5)  = u_j^n - \ldots \\
        & \frac{\sigma}{2}\left( 2\Delta xu_x - \frac{2}{3}\Delta x^3 u_{xxx} + \frac{1}{2}\Delta x^4u_{x^{(4)}} + \mathcal{O}(\Delta x^5) \right) + \ldots \\
        & \frac{\sigma^2}{2}\left( \Delta x^2 u_{xx} - \Delta x^3 u_{xxx} + \frac{7}{12}\Delta x^4 u_{x^{(4)}} - \frac{1}{4}\Delta x^5u_{x^{(5)}} + \mathcal{O}(\Delta x^6) \right)\\
        & = u_j^n - \sigma\Delta xu_x + \frac{\sigma^2}{2}\Delta x^2u_{xx} - \frac{1}{6}(3\sigma^2 - 2\sigma)\Delta x^3u_{xxx} + \frac{1}{24}\left(7\sigma^2 - 6\sigma\right)\Delta x^4u_{x^{(4)}}
    \end{align*}

    \pagebreak

    Starting with subtracting the $u_j^n$ terms and expanding $\sigma$ gives,

    \vspace{-0.35in}
    \begin{align*}
        \Delta tu_t + \frac{1}{2}\Delta t^2u_{tt} + & \frac{1}{6}\Delta t^3 u_{ttt} + \frac{1}{24}\Delta t^4u_{t^{(4)}} + \ldots \mathcal{O}(\Delta t^5) \\
        & = \frac{a\Delta t}{\Delta x}\left( - \Delta xu_x + \frac{\sigma}{2}\Delta x^2u_{xx} - \frac{1}{6}(3\sigma - 2)\Delta x^3u_{xxx} + \frac{1}{24}\left(7\sigma - 6\right)\Delta x^4u_{x^{(4)}} \right)\\
        \shortintertext{From here I will simplify by dividing through by $\Delta t$ and distributing $\Delta x$,}
        u_t + \frac{1}{2}\Delta tu_{tt} + & \frac{1}{6}\Delta t^2 u_{ttt} + \frac{1}{24}\Delta t^3u_{t^{(4)}} + \ldots \mathcal{O}(\Delta t^4) \\
        & = a\left(- u_x + \frac{\sigma}{2}\Delta xu_{xx} + \frac{a}{6}(3\sigma - 2)\Delta x^2u_{xxx} + \frac{1}{24}\left(7\sigma - 6\right)\Delta x^4u_{x^{(4)}}\right)\\
        \shortintertext{Collecting the one-dimensional advection term to the same side,}
        u_t + au_x & = -\frac{1}{2}\Delta tu_{tt} - \frac{1}{6}\Delta t^2 u_{ttt} - \frac{1}{24}\Delta t^3u_{t^{(4)}} + \frac{\sigma a}{2}\Delta xu_{xx} + \ldots \\
        & + \frac{a}{6}(3\sigma - 2)\Delta x^2u_{xxx} + \frac{1}{24}\left(7\sigma - 6\right)\Delta x^4u_{x^{(4)}}\\
        \shortintertext{Now with the expression for the one-dimensional advection solved for, I will relate temporal derivatives to spatial indices by conducting expansions,}
        u_{tt} & = -\frac{1}{2}\Delta t u_{ttt} - au_{xt} + \frac{\sigma a}{2}\Delta xu_{xxt} + \mathcal{O}(\Delta x^2, \Delta t^2)\\
        u_{tx} & = -\frac{1}{2}\Delta t u_{ttx} - au_{xx} + \frac{\sigma a}{2}\Delta xu_{xxx} + \mathcal{O}(\Delta x^2, \Delta t^2)\\
        u_{ttt} & = -au_{xtt} + \mathcal{O}(\Delta x, \Delta t)\\
        u_{txx} & = -au_{xxx} + \mathcal{O}(\Delta x, \Delta t)\\
        u_{ttx} & = -au_{xxt}\\
        \shortintertext{With the higher mixed-derivatives solved for, backtracking will find the $\alpha$ and $\beta$ coefficients,}
        u_{txx} & = -au_{xxx}\\
        u_{ttx} & = a^2u_{xxx}\\
        u_{ttt} & = -a^3u_{xxx}\\
        u_{tx} & = -\frac{1}{2}\Delta t a^2 u_{xxx} - au_{xx} + \frac{\sigma a}{2}\Delta x u_{xxx}\\
            & = -au_{xx} + \cancelto{0}{\left(\frac{\sigma a}{2}\Delta x - \frac{a^2\Delta t}{2}\right)}u_{xxx} = -au_{xx}\\
        u_{tt} & = \frac{1}{2}\Delta t(a^3u_{xxx}) + a^2u_{xx} - \frac{\sigma a^2}{2}\Delta xu_{xxx} = a^2u_{xx}\\
        u_t + au_x & = -\frac{1}{2}\Delta ta^2u_{xx} + \frac{1}{6}\Delta t^2 a^3u_{xxx} + \frac{\sigma a}{2}\Delta xu_{xx} + \frac{a}{6}(3\sigma - 2)\Delta x^2u_{xxx} + \mathcal{O}(\Delta x^3, \Delta t^3)\\
        & = \underbrace{\cancelto{0}{\left(-\frac{1}{2}\Delta ta^2  + \frac{\sigma a}{2}\Delta x\right)}}_{\alpha}u_{xx}  +  \underbrace{\left( \frac{1}{6}\Delta t^2 a^3 + \frac{a}{6}(3\sigma - 2)\Delta x^2\right)}_{-\beta}u_{xxx}\\
        & = 0\cdot u_{xx} + a\left(\frac{1}{6}\frac{\Delta x^2}{\Delta x^2}\Delta t^2a^2 + \frac{a}{6}(3\sigma - 2)\Delta x^2\right)\\
        & = 0\cdot u_{xx} + a\left(\frac{\Delta x^2}{6}\sigma^2 + \frac{a}{6}(3\sigma - 2)\Delta x^2\right)u_{xxx}\\
    \end{align*}

    After further simplifications,

    \vspace{-0.35in}
    \begin{align*}
        u_t + au_x & = 0\cdot u_{xx} + \frac{a\Delta x^2}{6}\left(\sigma^2 - 3\sigma +2\right)u_{xxx}
        \shortintertext{This gives that the $\alpha$ and $\beta$ expressions are,}
    \end{align*}

    \vspace{-0.5in}
    \begin{equation*}
        \boxed{\alpha = 0,\quad \beta = -\frac{a\Delta x^2}{6}(\sigma^2 - 3\sigma + 2)}
    \end{equation*}

    \begin{fminipage}{0.9\linewidth}
        \textbf{Looking above to the dispersion (the coefficient $\bf \beta$) will denote how waves of different frequencies will move at different speeds. This dispersion term will be the cause of oscillations where they were not present before.}
    \end{fminipage}

    \item Perform a von-Neumann stability analysis of the Beam-Warming method.  What is the stability limit for the CFL number $\sigma$?
    
    \vspace{-0.35in}
    \begin{align*}
        \shortintertext{In order to complete the von-Neumann stability analysis, first look to the $\beta$ coefficient to be greater than or equal to zero as the limiting case,}
        \sigma^2 - 3\sigma + 2 & \ge 0\\
        \shortintertext{Performing simple factorization,}
        (\sigma - 2 )(\sigma -1) & \ge 0\\
        \shortintertext{This gives the roots to be,}
        \sigma & = 1,\ 2\\
        \shortintertext{Since these are the roots, and this is a concave-up parabolic function then from $\sigma \in [1,2] < 0$ thus the actual limits for the CFL number $\sigma$ as,}
    \end{align*}

    \vspace{-0.5in}
    \begin{equation*}
        \boxed{\sigma \in [0,1] \cup [2, \infty)}
    \end{equation*}

    \item Implement the BW method in a computer program using $L= 2,\ a= 0.5,\ u_0(x) = exp[-100(x/L-0.5)^2]$ and a final time of $T=L/a$ (1 period).  Perform spatial and temporal convergence studies to demonstrate the order of accuracy in space and time.
\end{enumerate}